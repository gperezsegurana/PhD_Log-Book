\labday{Monday, 7th January 2019}

Lectures, 2 on Short wavelenght accelerators, 1 on MAD-X.\\
Add \verb|Needs["SDDS`"]| to mathematica initialization to handle SDDS files generated with \verb|sdds2math|\\


\labday{Tuesday, 8th January 2019}

\experiment{Meeting with Pete}
\begin{itemize}
    \item Write up a few pages on work done analyzing MAX-IV BC1 \& BC2
    \item Take a look at literature on LiTrack by Paul Emma @ SLAC (Download \& Play)
    \item Study Peter's MMA notebooks on 1D tracking for PERLE \& UK-XFEL
    \item Read about short-range wakefields
    \item Learn from Andy Wolski notes \& book about ISR
    \item Ask for Matthew Sands SLAC note
\end{itemize}
Started writing ``report'' on MAX-IV BC1\\

\labday{Wednesday, 9th January 2019}
Write up of ``report'' on MAX-IV BC1\\
Setup MatLab with LiTrack\\
Got A. Wolski Book and notes\\


\labday{Thursday, 10th January 2019}
Reading Effects of Synchrotron Radiation (Chapter 7 of A. Wolski)\\


\labday{Friday, 11th January 2019}
Finished and wrote up analysis of MAZ-IV BC2\\
Explored Pete's MMA traking notebooks\\
Analysis of Jonas' BC1 Lattice\\

\labday{Monday, 14th January 2019}
Lectures. 2h on Short Wavelength Accelerators, 1 on MAD-X\\
Analysis of Jonas' Max4 lattices\\


\labday{Tuesday, 15th January 2019}
Improved flexibility of elegant scripts and MMA analysis notebooks\\
Started dissecting Igor's paper on \emph{Semi-analytical modeling of multistage bunch compression with collective effects}\\

\labday{Wednesday, 16th January 2019}
\emph{Semi-analytical modeling of multistage bunch compression with collective effects}\\

\labday{Thursday, 17th January 2019}
\emph{Semi-analytical modeling of multistage bunch compression with collective effects}\\
Investigated design options to obtain a variable $R_{56}$ in Max-4 BC. \\
\begin{itemize}
    \item Add Quads such that dispersion is 0 at dipoles
    \item Add Reverse bends to cancel $R_{56}$
\end{itemize}
\experiment{UK-XFEL WP3 meeting}

\labday{Friday, 18th January 2019}
\experiment{Meeting with Pete}
Checked conceptual design changes to Max-4 BC\\
Will investigate both cases altering the BeamLines with Mathematica and then generating Elegant and Mad8dl lattices to run simulations\\
Once results are obtained from simulations, check:
\begin{enumerate}
    \item Is the new lattice \textbf{isochronus}?
    \item Is the new lattice physical? (Non ridiculous 20T fields or overlaping elements)
    \item Is it possible to have a continuous change of $R_{56}$ from current value to 0?
\end{enumerate}
Started playing with Mathematica in order to alter Lattices.

\labday{Monday, 21st January 2019}

Lectures. 3h on MAD-X\\

\labday{Tuesday, 22nd January 2019}

Re-designed Max4 BC1 with extra quads\\

\labday{Wednesday, 23rd January 2019}

Disecting Peter's examples using MMA to manipulate beamlines.\\
Minimized $R_{56}$ in Max4 BC1, but need to handle $y$ dispersion. (will add defocusing quads making the arcs DBAs)\\
A. Bainbridge seminar on the history of accelerator science and development at Daresbury\\


\labday{Wednesday, 30th January 2019}

\experiment{Meeting with Pete}
\begin{itemize}
    \item For additional quad scheme
    \begin{itemize}
        \item Can $R_{56}$ really be zero? (by miss-powering quads such that R56 from dispersion from within the dipoles cancels overall)
        \item Can we make a halfway house between $R_{56} = +30$mm and \~ zero
        \item Check dipole edge angles
        \item Put it into a presentation
        \item Reverse bend
    \end{itemize}
\end{itemize}