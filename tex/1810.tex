% Blank template to use for new days:

%\labday{Day, Date Month Year}

%\experiment{}

%Text

\labday{Monday, 1 October 2018}

\experiment{Meeting with Pete}
Went through induction.\\
Filed forms to set up IT account and access to resources.\\

%%%%%%%%%%%%%%%%%%%%%%%%%%%%%%%%%%%%%%%%%%%%%%%%%%%%%%%%%%%%%%

\labday{Tuesday, 2 October 2018}

\experiment{Meeting with Pete}
Setup Box and Mendeley groups to share publications and code.\\
Among the documents shared so far, note Pellegrini D. thesis.\\
Also document that justifies my studentship as it should serve as a general guideline of the whole PhD.\\
In the process of setting up a Unix account and VPN access to DL.\\

\labday{Wednesday, 3 October 2018}

@Lancaster\\
Registration\\
PhD Induction and FST induction\\
\experiment{Meeting with Ian}
Discussed practical arrangements of supervision (Pete day to day and Ian Admin)\\
PGTA wont be happening Michaelmas term 2018 since CAS school is priority. Maybe in the future if logistics work.\\
Mini-lecture on transverse and longitudinal linear beam dynamics. Pointed towards CI library for introductory books with problems and solutions\\
Will see again on Friday after department induction to try figure out how the payment system works for travel expenses.\\

\labday{Thursday, 4 October 2018}
Finished \emph{Introduction to Beam Dynamics} lectures on CI archive. Started RF.


\labday{Friday, 5 October 2018}
@Lancaster\\
Completed department PhD induction. Returned dibber.
\experiment{Meeting with Ian}
Discussed possibility of PGTA for 353 and other. For now, put down as backup.\\
Will go forward with making the payment for CAS school and later claim the cost back when the system to do so is up.\\




\labday{Monday, 8 October 2018}

Reviewed first two lectures of RF for linacs on the archive.\\
Introduction talk and meet \& greet with the rest of PhD students.\\

\labday{Tuesday, 9 October 2018}

\experiment{Meeting with Pete}
Discussed \textbf{Bunch compression} and the importance of it for ERLs\\
To drive FEL or Compton source need SMALL "longitudinal emittance"\\
Relation between adiabatic damping and acceleration\\
$R_{56}$ ``momentum compaction" and sign conventions \\
4-Dipole bunch compressor ``CHICANE-LIKE COMPRESSION''\\
Triple-Bend achromat arc ``ARC-LIKE COMPRESSION'' and the effects the strength of the quadrupoles have on the sign of $R_{56}$ and how we can set it's $R_{56}$\\

\labday{Wednesday, 10 October 2018}
1st half of the \emph{Transverse linear beam dynamics} course from The 2018 introduction to accelerator physics school

\labday{Thursday, 11 October 2018}

2nd half of the \emph{Transverse linear beam dynamics} and \emph{Particle Motion in Hamiltonian Formalism} course from the 2018 introduction to accelerator physics school.


\labday{Friday, 12 October 2018}

Installed Elegant!


\labday{Monday, 14 October 2018}

First set of lectures at CI. 1 hour on Intro to Acc. Physics and 2 on special relativity and EM.\\
Things to look into:
\begin{itemize}
    \item 9 Tesla Cyclotron. LHC does not have the biggest magnet
    \item Betatrons
    \item John Blewet
    \item Nick Christofolus (strong focusing patent)
    \item Renate Chasman (Achromats) Phys. prospect. 10(2008)438\\ \HRule
    \item Vlosov models $f(t,\textbf{p},\textbf{v})$ vs. fluid models $\textbf{v}(\textbf{v},t)$
\end{itemize}


\labday{Tuesday, 15 October 2018}

RTF Access training


\labday{Wednesday, 16 October 2018}

Fire \& Safety training


\labday{Thursday, 17 October 2018}

Set bash and vim to usable form.\\
Bash looks like Ubuntu defaults.\\
Vim has murphy color scheme.\\
Set Xming as a Xserver to have a window environment with WSL.\\
UK-XFEL Quarterly meeting.


\labday{Friday, 18 October 2018}

\experiment{Things for Gus to be getting on with. by Peter Williams}
\begin{itemize}
\item Read all material + run elegant on ER@CEBAF
\item Max-IV
\begin{itemize}

    \item \todo[inline]{Make $R_{56}$ variable \textbf{down to zero}}
    \item \todo[inline]{analyze this system $$B_{x,y}, D_{x,y}, \mu_{x,y}, R_{56}, T_{566}, W_{x}, \frac{dD_{x}}{d\delta}$$}
    \item \todo[inline]{Can it be made \textbf{apochromatic}? }
ie. $ W_{x,y}=0 $ ... but what about $\frac{dD_{x}}{d\delta}$ and $T_{566}$
\end{itemize}
\end{itemize}



\labday{Monday, 22 October 2018}



\labday{Tuesday, 23 October 2018}



\labday{Wednesday, 24 October 2018}



\labday{Thursday, 25 October 2018}



\labday{Friday, 26 October 2018}